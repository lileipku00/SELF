\documentclass[12pt]{softwaremanual}


\author{Joseph Schoonover}
\title{}
\date{}


\begin{document}

% Doing a custom title-page
\begin{titlingpage}
    
        \vspace*{2cm}
        
     \begin{flushright}
        {\fontfamily{cmss}\selectfont
        \HUGE{\textbf{ Spectral Element Libraries in Fortran }}\\
        }
       
        \vspace{1cm}
        
        \huge{
        \textbf{
        \textit{
        \textcolor{blue}{
           SELF
        }}}}
        
     \end{flushright}
         
        \vspace{2cm}
        
     \begin{center}
     
        %Do a subtitle here if you like
        {\fontfamily{cmss}\selectfont
        \huge{
           Quick Reference Manual
        }
        
        \vspace{1.5cm}
        
        % Enter the author's name
        \textbf{
        \large{
           \theauthor 
         }}}
        
        \vfill
        
        
        \vspace{0.8cm}
        
     \end{center}
        
    
\end{titlingpage}



\tableofcontents

\pagestyle{myheadings}
\chapter{Software Overview}
The Spectral Element Libraries in Fortran (SELF) provide supporting data-structures for implementing spectral element methods (SEMs) to solve partial differential equations (PDEs) in multiple dimensions. The focus of the libraries is particularly on Legendre-Galerkin type methods. Legendre-Galerkin methods solve PDEs by approximating the solution and geometry by interpolating polynomials. Discrete equations are formed by solving the weak form of the PDE in which the integrals are approximated by discrete Gauss or Gauss-Lobatto quadrature. Underneath the umbrella of ``Legendre-Galerkin'' are Continuous Galerkin (CG) and Discontinuous Galerkin (DG) methods. CG methods are primarily used for elliptic or parabolic PDEs while DG methods are focused on hyperbolic systems.

The software is broken into the following components :
\begin{enumerate}
\item Commonly used routines and dictionaries,
\item Interpolation and Quadrature,
\item Spectral Operator storage structures,
\item Spectral filters,
\item Solution storage data structures,
\item Geometry,
\item \textbf{\textit{High end solvers}}
\end{enumerate}

This document primarily focuses on the use and modification of the pre-constructed and tested high-end solvers. The first six components all serve as support for computing high order discrete diferentiation on structured or unstructured meshes. Modification of these lower-end support modules is not recommended. 

\chapter{Getting Started}
The quickest way to get started is to browse through the examples subdirectory and identify a test-case that is appealing to you. The next steps are to compile and execute the example program. This process is now described for the ``spot-advection'' example for for advection3d highend solver.




\end{document}